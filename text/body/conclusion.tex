\section{Conclusion} \label{sec:conclusion}

Mutations are fundamental drivers of evolutionary processes, generating the genetic diversity necessary for populations to adapt and survive \citep{hershberg2015mutation}.
% Developing this understanding is important Understanding the factors that influence selection on mutation rates themselves is therefore key in understanding and managing evolutionary dynamics within asexual populations.
Ongoing efforts have established an increasingly detailed theoretical basis to explain mutator fixation within large populations, by way of hitchhiking on discovery of beneficial mutations \citep{raynes2011mutator,raynes2013effect,raynes2018sign,raynes2019selection,raynes2019migration}.
However, less understanding exists as to factors suppress increases in mutation rate within very large asexual populations undergoing adaptive evolution.
In this work, we have applied simulation experiments to characterize the role of fitness landscape adaptive potential in this knowledge gap.

% This work contributes experiments to better characterize factors affecting selection on mutator traits in very large populations.
% replicated the sign-change effect of population size on mutator fixation probability, and
In a first set of experiments, we confirmed that, under the assumption of unlimited adaptive potential, increases in population size transition outcomes to a regime where mutators become strongly favored.
In line with expectations, mutators reliably fixed across population sizes up to a billion agents — substantially larger than had been previously tested using agent-based simulation.
However, restricting adaptive potential reversed outcomes in very large populations.
When limited adaptive mutations were available to discover, a tertiary regime appears where nonmutators regain selective favor within very large populations.
This effect appeared consistently across both a counter-based genome model, adapted from \citet{raynes2018sign}, and a more literal ``site-explicit'' genome model where alleles with potential to experience adaptive mutation were tracked directly.

Subsequent experiments investigated the influence of population structure and population composition on adaptation-constrained mutator dynamics.
We found outcomes to be highly sensitive to baseline frequency of mutators.
When mutators were initially rare, mutator fixation was reliably suppressed in very large populations through substantially more available adaptive mutations.
Strong, highly-localized spatial structure further amplified nonmutator resilience in very large populations, but only when mutators were initially rare.
Notably, though, well-mixed populations with weak spatial structure experienced opposite effects of population size on mutator dynamics.
Rather than stabilizing against mutator invasion, mutators more readily fixed under well-mixed conditions.

Much future work remains.
One phenomenon that has been observed in benchtop evolution experiments \citep{ho2021evolutionary}, but is not considered in present work, is mutator reversion.
In this scenario, gain-of-function mutations restore a mutator strain toward baseline nonmutator rates, either partially or fully.
Crucially, reversion events introduce the possibility of escape from otherwise permanent mutator fixation \citep{taddei1997role}.
Indeed, even low-frequency recobmination processes that faciliate restoration of DNA repair mechanisms can strongly influence the fate of mutators \citep{tenaillon2000mutators}.
Additionally, even within model systems operating within an asexual regime, strain ploidy has been shown to strongly influence mutator dynamics \citep{thompson2006ploidy}.

Given experimental evidence highlighting limitations of mutator strategies in facilitating adaptation within complex environments \citep{ho2021evolutionary}, another informative extension would be to explore how mutator dynamics manifest within richer fitness landscapes that capture aspects of the pleiotropy, redundancy, asymmetries characteristic in biology.
Within this framework, it would be possible to study adaptive potential in terms of stressors or environmental changes that distort fitness landscape topology, rather than artificial caps on available beneficial mutations.
Testing models of adaptive potential with an asymptotic supply of beneficial mutations (rather than simply capped) would also be informative.

Use of hardware accelerator technology for high-performance computing was key in enabling agent-based simulation at the population scales explored in this work.
However, further work remains in algorithm and software development to leverage these platforms for \textit{in silico} evolution experiments.
For example, although trivial for GPU architecture, implementing well-mixed (or nearly well-mixed) migration on WSE hardware will necessitate densely-connected, hierarchical, or fully point-to-point routing structures among processor elements \citep{james2020physical,luczynski2024near}.
Across all future experiments, insight could be deepened by tracking phylodynamics within simulated populations.
Recent work has demonstrated potential for lightweight reconstruction-based strategies for phylogenetic tracking in many-processor evolution simulations \citep{moreno2022hereditary}, important methods development work remains in providing sufficient tree reconstruction throughput to enable large-scale analyses \citep{moreno2024trackable}.
% Such an approach might ultimately incorporate implementation for runtime sampling of genomes from device to host, to serve as ``fossils'' in revealing intermediate phenotypic characters and fleshing out the lineage structure of extinct clades \citep{moreno2024guide}.

Findings from this work highlight the role of available adaptive potential, in conjunction with spatial structure and background mutator frequency, in influencing the evolution of mutation rates in asexual populations.
In concert with empirical data from \textit{in vivo} experiments, simulation studies are key in guiding derivation of formulae describing relationships among evolutionary dynamics — as has been accomplished in earlier model-based investigations of mutator dynamics \citep{wylie2009fixation,raynes2018sign}.
In the case of present work, such formulations seem likely to largely revolve around projections of mutation waiting times and time to fixation for beneficial alleles \citep{ribeck2016competition}.
Further, in the face of scenarios involving extensive mechanistic complications, agent-based simulations can supplement mathematical approaches as predictive instruments in their own right \citep{an2009agent}.
To this end, we see work increasing feasible scale of simulation by leveraging accelerator hardware as a key contributing step in achieving predictive capabilities more directly representative of real-world biological systems.
Ultimately, very large-scale simulations may drive so-called ``digital twins'' of evolving populations --- high-fidelity computational models that incorporate real-time feedback to mirror their physical counterpart \citep{dekoning2023digital}.
Such digital twins will not only advance our theoretical insights but also provide practical tools for predicting and controlling evolutionary trajectories in natural and artificial populations.
We expect that developments in this vein will play a role in addressing complex biological challenges in medicine, public health, and natural resources management, as well as enriching our understanding of evolution in an ever-changing world.
