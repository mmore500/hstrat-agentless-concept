\section{Conclusion} \label{sec:conclusion}

Mutations are fundamental drivers of evolutionary processes, generating the genetic diversity necessary for populations to adapt and survive \citep{hershberg2015mutation}.
% Developing this understanding is important Understanding the factors that influence selection on mutation rates themselves is therefore key in understanding and managing evolutionary dynamics within asexual populations.
Ongoing efforts have established an increasingly detailed theoretical basis to explain hypermutator fixation within large populations, by way of hitchhiking on discovery of beneficial mutations \citep{TODO}.
However, less understanding exists as to factors suppress increases in mutation rate within very large asexual populations undergoing adaptive evolution.
In this work, we have aplied simulatoin exeriments to charactreize the role of fitness landscape adpative potential in this knowledge gap.

% This work contributes experiments to better characterize factors affecting selection on hypermutator traits in very large populations.
% replicated the sign-change effect of population size on hypermutator fixation probability, and
In a first set of experiments, we confirmed that, under the assumption of unlimited adaptive potential, increases in population size transition outcomes to a regime where hypermutators become strongly favored.
In line with expections, hypermutation reliably fixed through population sizes up to a billion agents --- substantially larger than had been previously tested using agent-based simulation.
However, restricting adaptive potential reversed outcomes in very large populations.
When limited adaptive mutations were available to discover, a tertiary regime appears where normomutators regain selective favor within very large populations.
This effect appeared consistently across both a counter-based genome model, adapted from \citet{raynes2018sign}, and a more literal ``site-explicit'' genome model where alleles with potential to experience adaptive mutation were tracked directly.

Subesquent experiments investigated the influence of population structure and population composition on adaptation-constrained hypermutator dynamics.
We found outcomes to be highly sensitive to baseline frequency of hypermutators.
When hypermutators were initially rare, hypermutator fixation was reliably suppressed in very large populations through substantially more available adaptive mutations.
Strong, highly-localized spatial structure further amplified normomutator resilience in very large populations, but only when hypermutators were initially rare.
Notably, though, well-mixed populations with weak spatial structure experienced opposite effects of population size on hypermutator dynamics.
Rather than stabilizing against hypermutator invasion, hypermutators more readily fixed under well-mixed conditions.

Much future work remains.
One phenomenon that has been observed in benchtop evolution experiments \citep{ho2021evolutionary}, but we have not yet considered in present work, is hypermutator reversion.
In this scenario, gain-of-function mutations restore a hypermutator strain toward baseline normomutator rates, either partially or fully.
Crucially, reversion events introduce the possibility of escape from otherwise permanent hypermutator fixation.
Another informative extensions would be to explore how hypermutator dynamics manifest within richer fitness landscapes, such as the NK model, which captures aspects of the pleiotropy, redundancy, asymmetries, and dimensionality characteristic of natural organisms.
Within this framework, it would be possible to study adaptive potential in terms of stressors or environmental changes that distort fitness landscape topology, rather than artifical caps on available beneficial mutations.

HPC hardware accelerator technology was key in enabling agent-based simulation at the population scales explored in this work.
However, further work remains in algorithm and software development to leverage these platforms for \textit{in silico} evolution experiments.
For example, although trivial for GPU architecture, implementing well-mixed (or nearly well-mixed) migration on WSE will necessitate densely-connected, hierarchical, or fully point-to-point routing structures among processor elements \citep{james2020physical,luczynski2024near}.
Across all future experiments, insight could be deepened by supplementing end-state data with records tracing dynamics across the time course of evolution.
Capturing and exporting time series, though, will need to be balanced against limited onboard memory (about 48kb per PE) and finite interconnect bandwidth.
For this purpose, we intend to apply new algorithms for fixed-footprint, zero-overhead timeseries downsampling to dynamically curate observations strided across a time window of interest, such as until local fixation occurs \citep{moreno2024structured}.
Phylogenetic tracking could also provide useful insight into simulation dynamics.
Although recent work has demonstrated potential for lightweight reconstruction-based strategies for phylogenetic tracking in many-processor evolution simulations \citep{moreno2022hereditary}, further work remains in optimizing runtime overhead on hardware accelerator platforms and increasing throughput for \textit{post hoc} tree-reconstruction \citep{moreno2024trackable,moreno2024analysis}.
Such an approach might ultimately incorporate implementation for runtime sampling of genomes from device to host, to serve as ``fossils'' in fleshing intermediate phenotypic characters and the lineage structure of extinct clades \citep{moreno2024guide}.

Findings from this work highlight the role of available adaptive potential, in conjunction with spatial structure and background hypermutator frequency, in influencing the evolution of mutation rates in asexual population.
In concert with empirical data from \textit{in vivo} experiments, simulation studies are key in guiding derivation of formulae describing relationships among evolutionary dynamics --- as has been accomplished in earlier model-based investigations of hypermutator dynamics \citep{raynes2018sign,raynes2019migration,raynes2019selection}cd.
In the case of present work, such formulations seem likely to largel tex
y revolve around projections of mutation waiting times and time to fixation for beneficial alleles \citep{ribeck2016competition}.
Additionally, though, agent-based simulations can supplement mathematics as predictive instruments in their own right.
To this end, we see work increasing feasible scale of simulation by leveraging accelerator hardware as a key contributing step in achieving predictive capabilities more directly representative of real-world biological systems.
Ultimately, very large-scale simulations may drive so-called ``digital twins'' of evolving populations --- high-fidelity computational models that incorporate real-time feedback to mirror their physical counterpart \citep{dekoning2023digital}.
Such digital twins will not only advance our theoretical insights but also provide practical tools for predicting and controlling evolutionary trajectories in natural and artificial populations.
We expect ongoing work in this space one day assist in addressing complex biological challenges in medicine, public health, and natural resources management, as well as enriching our understanding of evolution in an ever-changing world.
