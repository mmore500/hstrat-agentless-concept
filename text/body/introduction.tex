\section{Introduction} \label{sec:introduction}

Mutation imposes a fundamental evolutionary trade-off on asexual organisms.
Because mutations are typically orders of magnitude more likely to reduce the fitness of offspring than to enhance it, minimizing mutation increases the proportion of offspring that are viable.
However, in rare instances, mutation introduces a trait that benefits viability rather than harming it.
Within asexual populations, beneficial mutations can set into motion a winner-takes-all scenario where all other lineages unable to achieve comparable fitness gains (whether through mutation or lateral gene transfer) are eventually driven to extinction.
Hence, mutation may also confer a rare --- but profound --- fitness benefit.

Like other phenotypic characteristics, the mutation rate of an organism can be strongly influenced by heritable traits --- for instance, by the efficacy of DNA repair mechanisms \citep{sniegowski2000evolution}.
In the absence of ongoing adaptive evolution, purifying selection will tend to suppress so-called ``hypermutator'' alleles that amplify mutation rate.
However, without a mechanism to ease linkage disequilibrium (e.g., recombination or lateral gene transer), hypermutator alleles can  ``hitch-hike'' when a fitness-advantaged strain sweeps to fixation and carries along its full genetic background in addition to any beneficial traits \citep{johnson1999beneficial}.
In fact, as hypermutation can systematically accelerate the pace of adaptive evolution \citep{orr2000rate}, hypermutator alleles may become \textit{more} likely to fix than their ``normomutator'' counterparts.

Notable recent work integrating \textit{in silico} agent-based modeling with \textit{in vivo} evolution experiments has pointed to the fact that large population size, in particular, can drive the probability of hypermutator traits fixing to near certainty \citep{raynes2018sign}.
In this work, we contribute to addressing a knowledge gap in understanding conditions that stabilize very large populations \textit{against} hypermutator fixation, focusing on the role of adaptive potential --- that is, the quantity of beneficial novel alleles available to discover via mutation.

\subsection{Causes and Consequences of Hypermutator Strains}

Hypermutator strains are observed across a wide variety of populations, including both natural systems and laboratory experiments \citep{sniegowski1997evolution,swings2017adaptive,maddamsetti2020divergent,cherry2018methylation}.
As such, a substantial body of literature has investigated the conditions under which hypermutator traits tend to arise.
As mentioned above, population size has been implicated as a key factor for hypermutator fixation, with the beneficial mutations necessary for hitch-hiking more likely to arise in large populations \citep{raynes2018sign}.
On the other hand, bottleneck events --- episodes that arbitrarily decimate a substantial portion of a population --- can suppress hypermutator fixation by disrupting propagation of the beneficial alleles that hypermutators hitchhike off of \citep{raynes2013effect}.
Population structure has also been found to influence mutator outcomes, with deviation from well-mixed conditions being linked to higher probability for hypermutator fixation so long as any connectivity remains between subpopulations \citep{raynes2019migration}.
In a vein paralleling the question of adaptive potential explored here, recent \textit{in vivo} experimental work has established evidence for connections between fitness landscape geometry and mutator outcomes.
In this work, \citet{callens2023hypermutator} conducted treatments exposing populations of \textit{E. coli} to either antibiotic or osmotic stressors, calibrated to comparable severities.
Because the chosen antibiotic agent targeted a pathway involving relatively few genes, fewer loci exhibited adaptive potential compared to the osmotic stress condition.
In line with the hypothesis explored via simulation experiments in this work, the high-adaptive-potential osmotic stressor induced significantly higher rates of hypermutator fixation.

Further progress in understanding how and why hypermutator traits fix has broad importance.
Hypermutator traits have been shown to broadly influence the balance between selection and random effects on genetic content, profoundly accelerating processes of genetic draft \citep{couce2017mutator}.
Hypermutator strains are observed among populations of pathogens arising in medical and public health contexts, where they can facilitate acquisition of novel traits such as antibiotic resistance, tumor progression, and changes in virulence \citep{eliopoulos2003hypermutation,jolivetgougeon2011bacterial,stern2016viral,schlesner2015hypermutation}.
For instance, emergence of hypermutator strains has been associated with treatment using antibiotics of last resort against opportunistic pathogens \citep{mehta2019essential}.
Thus, beyond theoretical interest, studying hypermutator dynamics also has concrete practical applications with real-world potential to benefit both individual health and public well-being.

\subsection{Major Results}

In our initial experiments, we confirmed --- consistent with expectations --- that large populations favor hypermutator fixation, under the assumption of unlimited adaptive potential.
However, restricting adaptive potential by limiting adaptive mutations available reversed outcomes in very large populations to favor normomutators.
Investigating the influence of population structure and composition, we found spatial structure to be protective against hypermutator fixation in very large populations, but only when hypermutators were initially rare.
Under well-mixed conditions, however, large populations instead facilitated hypermutator fixation under conditions with limited adaptive potential.

Of technical note, the large-scale simulations necessary for this work were made possible by exploiting hardware accelerator platforms --- specialized high-performance computing (HPC) peripherals capable of massive computational workloads.
In line with existing work applying Graphics Processing Units (GPUs) for agent-based modeling \citep{turpin2021xaevol,kosiachenko2019mass,perumalla2009switching,heinemann2007artificial,richmond2023flame}, we found a substantial performance boost: in our case, $378\times$ speedup over single-core CPU evaluation.
\footnote{%
Other notable approaches to scaling up evolution experiments have used more traditional cluster-based CPU modeling \citep{moreno2022best,collier2015large,ray1995proposal,turpin2020paevol} and sampling-based approximations \citep{taddei1997role}.
}
We also leveraged Wafer-Scale Engine (WSE) technology in this work, a cutting-edge AI/ML-oriented platform that delivers 850,000 processor cores packed together on a single chip.
Simulation on WSE achieved a further $294\times$ speedup, resulting in a net $111{,}091\times$ speedup from single-CPU evaluation.
These performance gains ultimately benefited the work by allowing us to employ forward-time agent-based simulation at full resolution across a variety of genome models and population structures.
