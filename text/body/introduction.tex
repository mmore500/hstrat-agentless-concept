\section{Introduction} \label{sec:introduction}

Recent advances in cellular automata have established rich frameworks for continuous-valued state spaces with flexible rule sets.
The Lenia project is particularly notable for the rich taxonomy of emergent behaviors that have been systematically and extensively characterized \citep{chan2020lenia,horibe2023exploring}.

Notable developments have also been made in incorporating differentiable frameworks to enable rule set development using gradient descent \citep{mordvintsev2020growing,hamon2022learning}.
Other approaches to discovering rule sets include evolutionary computation \citep{jain2024capturing}, quality-diversity search \citep{faldor2024toward}, and sampling from the representation space of LLMs (e.g., LLMs, computer vision, etc.) \citep{kumar2024automating}.
Rule sets may also be spatially localized within a cellular automata model and allowed to evolve in tandem with the state dynamics they enact within simulation \citep{plantec2023flowlenia}.

Pertinent references for cellular automata work within the realm of artificial life include,
\begin{enumerate}
\item Moveable Feast Machine \citep{ackley2023robust,ackley2019building,ackley2012movable},
\item Lenia \citep{chan2020lenia,chan2019lenia},
\item differentiable Lenia \citep{hamon2022learning},
\item particle Lenia \citep{mordvintsev2022particle},
\item flow Lenia \citep{plantec2023flowlenia},
\item neural cellular automata \citep{mordvintsev2020growing},
\item self-replicating neural cellular automata \citep{sinapayen2023selfreplication},
\item discretized differentiable cellular automata \citep{miotti2025differentiable}, and
\item earlier works include CAPOW \citep{griffeath2003new}, Larger than Life \citep{evans2001larger}, RealLife \citep{pivato2007reallife}, and SmoothLife \citep{rafler2011generalization}.
\end{enumerate}
In other areas of biological research, cellular automata approaches have also been applied in modeling physiological \citep{peak2004evidence,davidenko1992stationary}, ecological \citep{breckling2011cellular}, and social \citep{beltran2009forecasting} processes.

Although a major goal of work with cellular automata is as a platform to study eco-evolutionary processes, a key epistemological distinction of cellular automata approaches is that no explicit self-replicating unit is defined \citep{hamon2022learning}; instead, interest lies in the propagation of self-stabilizing ``soliton'' patterns in state space \citep{chan2019lenia}.
This enactivist approach contrasts agent-based, ``digital evolution'' approaches where an explicit self-replicating unit \citep{pennock2007models}.
Whereas such may be trivially tracked and characterized, implicit individuality introduces substantial challenges in tracking and characterizing eco-evolutionary dynamics within a simulation.

A key attractive aspect of cellular automata is in spatial locality of update rules, which makes evaluation well-suited to highly distributed processing \citep{ackley2023robust}.
Lenia, in particular, uses a local update rule based on element-wise convolution using fixed-size kernels.
Figure \ref{fig:lenia-pseudocode} provides pseudocode for the Lenia update function.
A nice visual overview of the Lenia update process can also be found at \url{https://developmentalsystems.org/sensorimotor-lenia/public/leniaVid.mp4}.

It is not hard to imagine how 2D or 3D lattice-based cellular automata with local update rules might effectively harness fabric-based hardware architectures such as the Cerebras Wafer-Scale Engine.
In fact, the Cerebras Software Development Kit (SDK) materials include an example implementation of Conway's Game of Life included with \citep{cerebras2024gol}.
Lenia, in particular, may potentially be compatible with an existing Field Equation modeling framework for the Wafer-Scale Engine \citep{woo2022disruptive}, which provides a higher-level, NumPy-like interface for programming on the Wafer-Scale Engine.
(Documentation for this project is hosted at \url{https://dirk-netl.github.io/WSE_FE/index.html}.)
Cerebras provides a PyTorch back-end, which may be helpful in working with neural Cellular Automata \citep{cerebras2022pytorch}.
Conveniently, reference code for differentiable Lenia used in prototype work for this project is implemented in PyTorch \citep{hamon2022learning}.

Past work with agent-based simulations on the Cerebras Wafer-Scale Engine have successfully applied hereditary stratigraphy methodology for decentralized phylogeny tracking \citep{moreno2024trackable}.
However, existing work leveraging this approach has assumed an explicit agent genome model and hooked into explicit agent replication events.
In this work, we explore how distributed phylogeny tracking via hereditary stratigraphy might be generalized to track interaction and/or replication dynamics in cellular automata-based simulations where explicit agent genomes and agent replication do not exist.

\subsection{Proposed Approach}

\begin{figure}
\includegraphics[width=0.8\textwidth]{img/hstrat-agentless-concept-schematic}
\caption{Proposed framework for incorporating distributed phylogeny tracking into CA systems.}
\label{fig:hstrat-agentless-concept}
\end{figure}


The proposed approach annotates each CA cell with dstream/hstrat annotation data.
(As per usual, dstream/hstrat annotations act as neutral instrumentation that do not influence simulation state.)
These annotations step forward synchronously with each CA update performed, meaning that semantic ``generations'' correspond to simulation timesteps.
Each update, dstream/hstrat annotations are ``propagated'' (copied) from current-timestep to next-timestep cells.
For this purpose, each timestep, an \textbf{attribution function} is calculated to generate a mapping from next-timestep cells within $A_{t+1}$ to current-timestep cells within $A_{t}$.
Next-timestep cells within $A_{t+1}$ then ``pull'' forward annotations from the current-timestep cell within $A_{t}$ they are attributed to.
For non-surjective attribution functions (i.e., where more than one next-timestep cell may be attributed to a single current-timestep cell), this propagation process naturally gives rise to a branching ``attribution tree'' structure.

Figure \ref{fig:hstrat-agentless-concept} provides a schematic overview of the proposed strategy for cellular automata phylogeny tracking.
Several desirable properties are worth noting,
\begin{itemize}
\item bulk updating many hstrat/dstream buffers synchronized to the same timestep is efficient (i.e., the $k(t)$'th bit of all items are randomized) and can be achieved through vectorized/matrix operations (e.g., NumPy, CuPy, JAX, etc.),
\item tracking data may be stored, managed, and transported directly through existing tensor/matrix frameworks (e.g., as u64 values),
\item tracking behavior can be configured in a modular/extensible manner by swapping attribution functions, and
\item in principle, it may be possible to design attribution functions robustly agnostic to unexpected/novel emergent soliton characteristics or behavior.
\end{itemize}
Additionally, existing desirable properties of hstrat/dstream-based tracking are preserved
\begin{itemize}
\item memory usage is fixed and known \textit{a priori},
\item tracking operations are fully decentralized,
\item tracking is robust to data loss, and
\item the methodology is well-suited to sampling-based data collection strategies.
\end{itemize}

Desirable properties of the attribution function and how such an attribution function should be formulated remain open questions.
In seeking to track attribution trees that correspond to soliton phylogeny, one possible framing is to pursue a coalescent-like property, that attribution lineages for cells within the same soliton traced back in time should tend to rapidly converge to a ``common ancestor.''
It also may be desirable that attribution functions should avoid traces from one soliton back into another (i.e., to prevent ``contamination'' from non-reactive collisions between solitons).
In formulating an attribution function to track soliton phylogenies, it may be acceptable to act inertly over parts of a soliton's spatial footprint, so long as useful behavior occurs reliably within spatio-temporally coherent regions (e.g., regions of peak state density).

It is possible that ``flawed'' attribution trees only approximately corresponding to soliton phylogeny (or exhibiting some other semantic tracking property entirely) may still generate useful information about cellular automata dynamics and/or history.
In this vein, more general applications for attribution trees could arise in tracking history within simulations modeling non-replicator/non-evolutionary systems.

The remainder of this section is organized into five further subsections, brainstorming possible project directions with respect to
\begin{enumerate}
\item capabilities to test,
\item ways to test,
\item configurations to test,
\item possible extensions, and
\item strategy limitations.
\end{enumerate}

Section \ref{sec:results} reports preliminary proof-of-concept work using a differentiable Lenia configuration exhibiting gliding solitons capable of autocatalytic self-replication \citep{hamon2022learning}.
This preliminary work, applies a naive local-maximum attribution function with a manually-chosen radius.
