\section{Results and Discussion} \label{sec:results}

\subsection{Sign-change Effects of Population Size}

Our first set of experiments sought to investigate how acted in larger populations than had been tested in original experiments
Previous work by \citet{raynes2018sign} has shown that as population size scales, selection on hypermutators can move through several regimes.
It starts out as neutral when population size is very small, owing to the power of stochastic effects.
As population size increases, hypermutators first become disfavored by selection.
Intuitively, this can be imagined as the population size becoming large enough to select against the mutation load imposed by hypermutator traits.
However, as population size increases even further, hypermutators can become favored.
This is because in large enough populations, the chance that at least one hypermutator will discover a beneficial mutation that allows it to sweep to fixation becomes a good chance.

With larger population count per PE, resolution of our population size sweep becomes too coarse grained to reflect this effect at small population sizes.

\begin{figure}[h]
  % adapted from https://tex.stackexchange.com/a/122813/316176
  \captionsetup[subfigure]{justification=raggedright}
  \begin{minipage}{0.7\textwidth}

    \begin{minipage}{0.04\textwidth}~\end{minipage}%
    \begin{minipage}{0.44\textwidth}
      \centering
      \itshape
      \textbf{unlimited} beneficial mutations
    \end{minipage}%
    \begin{minipage}{0.34\textwidth}
      \centering
      \itshape
      \textbf{one} beneficial mutation
    \end{minipage}

    ~\vspace{-1.5ex}

    % Top subfigure
    \begin{subfigure}[b]{\linewidth}
      \begin{minipage}{0.88\textwidth}
        \begin{minipage}{0.53\textwidth}
          \includegraphics[height=2.9cm, trim={0cm 0.2cm 3.9cm 0.8cm}, clip]{binder/binder-wse-5050-spatial2d-32atile-infben-traits.ipynb/binder/teeplots/wse-5050-spatial2d-32atile-infben-traits/errorbar=ci+hue=genotype+num-abm=(inf,)+style=genotype+viz=size-fixation-areaplot+x=population-size+y=fixation-probability+ext=.pdf}%
        \end{minipage}%
        \begin{minipage}{0.47\textwidth}
          \includegraphics[height=2.9cm, trim={1.3cm 0.2cm 3.9cm 0.8cm}, clip]{binder/binder-wse-5050-spatial2d-32atile-infben-traits.ipynb/binder/teeplots/wse-5050-spatial2d-32atile-infben-traits/errorbar=ci+hue=genotype+num-abm=(1.0,)+style=genotype+viz=size-fixation-areaplot+x=population-size+y=fixation-probability+ext=.pdf}
        \end{minipage}
      \end{minipage}%
      \hspace{-3ex}%
      \begin{minipage}{0.12\textwidth}
        \raggedright
        \caption{\footnotesize 32 agents per PE\\~\\~\\}
        \label{fig:wse-inf-one:32}
      \end{minipage}%
    \end{subfigure}%

    % Bottom subfigure - Adjusted layout to match the top
    \begin{subfigure}[b]{\linewidth}
      \begin{minipage}{0.88\textwidth}
        \begin{minipage}{0.53\textwidth}
          \includegraphics[height=2.9cm, trim={0cm 0.2cm 3.9cm 0.8cm}, clip]{binder/binder-wse-5050-spatial2d-2048atile-infben-traits.ipynb/binder/teeplots/wse-5050-spatial2d-2048atile-infben-traits/errorbar=ci+hue=genotype+num-abm=(inf,)+style=genotype+viz=size-fixation-areaplot+x=population-size+y=fixation-probability+ext=.pdf}%
        \end{minipage}%
        \begin{minipage}{0.47\textwidth}
          \includegraphics[height=2.9cm, trim={1.3cm 0.2cm 3.9cm 0.8cm}, clip]{binder/binder-wse-5050-spatial2d-2048atile-infben-traits.ipynb/binder/teeplots/wse-5050-spatial2d-2048atile-infben-traits/errorbar=ci+hue=genotype+num-abm=(1.0,)+style=genotype+viz=size-fixation-areaplot+x=population-size+y=fixation-probability+ext=.pdf}
        \end{minipage}
      \end{minipage}%
      \hspace{-3ex}%
      \begin{minipage}{0.1\textwidth}
        \caption{\footnotesize 2,048 agents per PE\\~\\~\\}
        \label{fig:wse-inf-one:2048}
      \end{minipage}%
    \end{subfigure}%

~\vspace{-1.2ex}

\includegraphics[width=\textwidth, trim={0cm 4.7cm 5.3cm 0cm}, clip]{binder/binder-wse-5050-spatial2d-2048atile-infben-traits.ipynb/binder/teeplots/wse-5050-spatial2d-2048atile-infben-traits/col=available-beneficial-mutations+errorbar=sd+hue=genotype+kind=line+style=genotype+viz=relplot+x=population-size+y=fixation-probability+ext=}

  \end{minipage}%
  \begin{minipage}{0.3\textwidth}
    \caption{%
      \textbf{Restricted beneficial mutation supply favors normomutators in large populations.}
      \footnotesize
      TODO, 50/50 start conditions on WSE.
    }
    \label{fig:wse-inf-one}
  \end{minipage}
\end{figure}


\subsection{Normomutators Gain Favor in Large Populations when Adaptive Potential is Limited}

\begin{figure*}

\begin{minipage}{0.65\textwidth}
  \includegraphics[width=\textwidth]{binder/binder-wse-5050-spatial2d-2048atile-traits.ipynb/binder/teeplots/wse-5050-spatial2d-2048atile-traits/col=available-beneficial-mutations+errorbar=ci+hue=genotype+style=genotype+viz=size-fixation-areaplot+x=population-size+y=fixation-probability+ext=.pdf}%
  \end{minipage}
\begin{minipage}{0.3\textwidth}
\caption{
\textbf{Availability of adaptive potential shifts favor from normomutators to hypermutators.}
\footnotesize
Area plots show normo- vs. hypermutator fixation probabilities.
As available beneficial mutations are increased, hypermutators gain favor in progressively larger population sizes.
Simulations were conducted on WSE using the Poisson genome model, with populations initialized to a 50/50 mix of normo- and hypermutators.
Subpopulations comprised 2,048 agents per PE.
Error bands show bootstrapped 95\% confidence intervals.
}
\end{minipage}

\end{figure*}


\subsection{Normomutators are Consistently More Resilient when Background Hypermutator Prevalence is Low}

\begin{figure}[h]
  % adapted from https://tex.stackexchange.com/a/122813/316176
  \captionsetup[subfigure]{justification=raggedright}
  \begin{minipage}{\textwidth}

    \begin{minipage}{0.1\textwidth}~\end{minipage}%
    \begin{minipage}{0.35\textwidth}
      \centering
      \itshape
      {\large
      \textbf{50/50} conditions
      }
    \end{minipage}%
    \begin{minipage}{0.45\textwidth}
      \centering
      \itshape
      {\large
      \textbf{de novo} conditions
      }
    \end{minipage}

    ~\vspace{-0.7ex}

    % Top subfigure
    \begin{subfigure}[b]{\linewidth}
        \begin{minipage}{0.42\textwidth}
          ~
          % TODO
          \includegraphics[height=5.5cm, trim={0cm 0.8cm 0cm 0cm}, clip]{binder/binder-wse-5050-spatial2d-traits.ipynb/binder/teeplots/wse-5050-spatial2d-traits/col=population-size+errorbar=ci+hue=genotype+layout=wide+viz=size-fixation-cliffplot+x=fixation-probability+y=available-beneficial-mutations+ext=.pdf}%
        \end{minipage}%
        \begin{minipage}{0.06\textwidth}
          ~
        \end{minipage}%
        \begin{minipage}{0.38\textwidth}
          ~
          % TODO
          \includegraphics[height=5.5cm, trim={1.2cm 0.8cm 0cm 0cm}, clip]{binder/binder-wse-denovo-spatial2d-traits.ipynb/binder/teeplots/wse-denovo-spatial2d-traits/col=population-size+col-label=+errorbar=ci+hue=genotype+layout=wide+viz=size-fixation-cliffplot+x=fixation-probability+y=available-beneficial-mutations+ext=.pdf}%
        \end{minipage}%
      \begin{minipage}{0.12\textwidth}
        \raggedright
        \large
        \caption{counter-based\\ genome model}
        \label{fig:denovo-5050-conditions:poisson}
        \vspace{20ex}
      \end{minipage}%
    \end{subfigure}%

    \vspace{-10ex}

    % Bottom subfigure
    \begin{subfigure}[b]{\linewidth}
        \begin{minipage}{0.41\textwidth}
          \includegraphics[height=5.5cm, trim={0cm 0.8cm 0cm 0cm}, clip]{binder/binder-wse-5050-spatial2d-explicitsite-traits.ipynb/binder/teeplots/wse-5050-spatial2d-explicitsite-traits/col=population-size+errorbar=ci+hue=genotype+layout=wide+viz=size-fixation-cliffplot+x=fixation-probability+y=available-beneficial-mutations+ext=.pdf}%
        \end{minipage}%
        \begin{minipage}{0.08\textwidth}
          ~
        \end{minipage}%
        \begin{minipage}{0.38\textwidth}
          \includegraphics[height=5.5cm, trim={1.2cm 0.8cm 0cm 0cm}, clip]{binder/binder-wse-denovo-spatial2d-explicitsite-traits.ipynb/binder/teeplots/wse-denovo-spatial2d-explicitsite-traits/col=population-size+col-label=+errorbar=ci+hue=genotype+layout=wide+viz=size-fixation-cliffplot+x=fixation-probability+y=available-beneficial-mutations+ext=.pdf}%
        \end{minipage}%
      \begin{minipage}{0.12\textwidth}
        \raggedright
        \large
        \vspace{10ex}
        \caption{site-explicit\\ genome model}
        \label{fig:denovo-5050-conditions:site-explicit}

        \includegraphics[height=5.5cm, trim={7.3cm 0cm 0cm 0cm}, clip]{binder/binder-cupy-denovo-spatial1d-traits.ipynb/binder/teeplots/cupy-denovo-spatial1d-traits/col=population-size+errorbar=ci+hue=genotype+layout=skinny+viz=size-fixation-cliffplot+x=fixation-probability+y=available-beneficial-mutations+ext=.pdf}%
      \end{minipage}%
    \end{subfigure}%

  \end{minipage}

  \vspace{-10ex}

  \begin{minipage}{\textwidth}
    \caption{%
      \textbf{Initial supply of mutators influences nonmutator resilience to available adaptive potential.}
      \footnotesize
      Areaplot strips track mutator fixation probability across surveyed levels of adaptive potential, with error bands providing bootstrapped 95\% confidence intervals.
      Strips are arranged left-to-right with ascending population size, indicated on top axis.
      In left column, populations are initialized with half of the agents as mutators.
      In right column, populations are initialized with all nonmutators; mutators must arise \textit{de novo}, with probability $10^{-6}$ per daughter.
      The \textit{de novo} regime boosts nonmutator resilience to adaptive potential, with large population sizes reliably resisting mutator fixation through upwards of 10 available beneficial mutations.
      Top panel \ref{fig:denovo-5050-conditions:poisson} gives results for the counter-based genome model.
      Bottom panel \ref{fig:denovo-5050-conditions:site-explicit} gives results under the site-explicit genome model, which are similar.
      Simulations were conducted on WSE, using a population size of 256 agents per PE.
      Supplementary Figure \ref{fig:fixheat-wse-256atile} details results in a tabular format.
    }
    \label{fig:denovo-5050-conditions}
  \end{minipage}
\end{figure}


We found this to be consistent across both the Poisson model and the site-explicit model.

\subseection{Population Structure determines the sign effect of population size}

\begin{figure}[h]
  % adapted from https://tex.stackexchange.com/a/122813/316176
  \captionsetup[subfigure]{justification=raggedright}
  \begin{minipage}{\textwidth}

    \begin{minipage}{0.1\textwidth}~\end{minipage}%
    \begin{minipage}{0.25\textwidth}
      \centering
      \itshape
      {\large
      \textbf{weak} population structure
      }
      (well-mixed)
    \end{minipage}%
    \begin{minipage}{0.25\textwidth}
      \centering
      \itshape
      {\huge
      $\longleftrightarrow$
      }

      (2D demes)
    \end{minipage}%
    \begin{minipage}{0.25\textwidth}
      \centering
      \itshape
      {\large
      \textbf{strong} population structure
      }

      (1D demes)
    \end{minipage}

    ~\vspace{-0.7ex}

    % Top subfigure
    \begin{subfigure}[b]{\linewidth}
        \begin{minipage}{0.3\textwidth}
          \includegraphics[height=5.2cm, trim={0cm 0.8cm 3.9cm 0cm}, clip]{binder/binder-cupy-5050-wellmixed-traits.ipynb/binder/teeplots/cupy-5050-wellmixed-traits/col=population-size+errorbar=ci+hue=genotype+layout=skinny+viz=size-fixation-cliffplot+x=fixation-probability+y=available-beneficial-mutations+ext=.pdf}%
        \end{minipage}%
        \begin{minipage}{0.06\textwidth}
          ~
        \end{minipage}%
        \begin{minipage}{0.26\textwidth}
          \includegraphics[height=5.2cm, trim={2.1cm 0.8cm 3.9cm 0cm}, clip]{binder/binder-cupy-5050-spatial2d-traits.ipynb/binder/teeplots/cupy-5050-spatial2d-traits/col=population-size+errorbar=ci+hue=genotype+layout=skinny+viz=size-fixation-cliffplot+x=fixation-probability+y=available-beneficial-mutations+ext=.pdf}%
        \end{minipage}%
        % \begin{minipage}{0.05\textwidth}
        %   ~
        % \end{minipage}%
        \begin{minipage}{0.25\textwidth}
          \includegraphics[height=5.2cm, trim={2.1cm 0.8cm 4.75cm 0cm}, clip]{binder/binder-cupy-5050-spatial1d-traits.ipynb/binder/teeplots/cupy-5050-spatial1d-traits/col=population-size+errorbar=ci+hue=genotype+layout=skinny+viz=size-fixation-cliffplot+x=fixation-probability+y=available-beneficial-mutations+ext=.pdf}%
        \end{minipage}%
      \begin{minipage}{0.12\textwidth}
        \raggedright
        \large
        \caption{50/50 initial conditions}
        \label{fig:spatial-structure:5050}
        \vspace{20ex}
      \end{minipage}%
    \end{subfigure}%

    \vspace{-10ex}

    % Bottom subfigure
    \begin{subfigure}[b]{\linewidth}
 \begin{minipage}{0.3\textwidth}
  \begin{tikzpicture}
    \node[anchor=south west, inner sep=0] (image) at (0,0) {
      \includegraphics[height=5.2cm, trim={0cm 0.8cm 3.9cm 0cm}, clip]{binder/binder-cupy-denovo-wellmixed-traits.ipynb/binder/teeplots/cupy-denovo-wellmixed-traits/col=population-size+errorbar=ci+hue=genotype+layout=skinny+viz=size-fixation-cliffplot+x=fixation-probability+y=available-beneficial-mutations+ext=.pdf}
    };
    \begin{scope}[x={(image.south east)}, y={(image.north west)}]
      \fill[white] (0.363, 0.27) rectangle (0.455, 0.82);  % censor
    \end{scope}
  \end{tikzpicture}
\end{minipage}
        \begin{minipage}{0.06\textwidth}
          ~
        \end{minipage}%
        \begin{minipage}{0.26\textwidth}
\begin{tikzpicture}
  \node[anchor=south west, inner sep=0] (image) at (0,0) {
    \includegraphics[height=5.2cm, trim={2.3cm 0.8cm 3.9cm 0cm}, clip]{binder/binder-cupy-denovo-spatial2d-traits.ipynb/binder/teeplots/cupy-denovo-spatial2d-traits/col=population-size+errorbar=ci+hue=genotype+layout=skinny+viz=size-fixation-cliffplot+x=fixation-probability+y=available-beneficial-mutations+ext=.pdf}%
  };
  \begin{scope}[x={(image.south east)}, y={(image.north west)}]
    \fill[white] (0.048, 0.27) rectangle (0.185, 0.82);  % censor
  \end{scope}
\end{tikzpicture}
        \end{minipage}%
        % \begin{minipage}{0.05\textwidth}
        %   ~
        % \end{minipage}%
        \begin{minipage}{0.25\textwidth}
\begin{tikzpicture}
  \node[anchor=south west, inner sep=0] (image) at (0,0) {
    \includegraphics[height=5.2cm, trim={2.3cm 0.8cm 4.7cm 0cm}, clip]{binder/binder-cupy-denovo-spatial1d-traits.ipynb/binder/teeplots/cupy-denovo-spatial1d-traits/col=population-size+errorbar=ci+hue=genotype+layout=skinny+viz=size-fixation-cliffplot+x=fixation-probability+y=available-beneficial-mutations+ext=.pdf}%
  };
  \begin{scope}[x={(image.south east)}, y={(image.north west)}]
    \fill[white] (0.057, 0.27) rectangle (0.225, 0.82);  % censor
  \end{scope}
\end{tikzpicture}
        \end{minipage}%
      \begin{minipage}{0.12\textwidth}
        \raggedright
        \large
        \vspace{10ex}
        \caption{\textit{de novo} initial conditions}
        \label{fig:spatial-structure:denovo}

        \includegraphics[height=5.5cm, trim={7.3cm 0cm 0cm 0cm}, clip]{binder/binder-cupy-denovo-spatial1d-traits.ipynb/binder/teeplots/cupy-denovo-spatial1d-traits/col=population-size+errorbar=ci+hue=genotype+layout=skinny+viz=size-fixation-cliffplot+x=fixation-probability+y=available-beneficial-mutations+ext=.pdf}%
      \end{minipage}%
    \end{subfigure}%

  \end{minipage}

  \vspace{-10ex}

  \begin{minipage}{\textwidth}
    \caption{%
      \textbf{Population structure increases nonmutator resilience to available adaptive potential.}
      \footnotesize
      Areaplot strips track mutator fixation probability across surveyed levels of adaptive potential, with error bands indicating bootstrapped 95\% confidence intervals.
      Strips are arranged left-to-right with ascending population size, marked on the top axis.
      Leftmost column shows well-mixed conditions.
      Populations in center and rightmost columns are subdivided into 256-agent demes, arranged in a 2D grid or 1D ring.
      Top panel, \ref{fig:spatial-structure:5050}, reports experiments initialized with an even mix of mutator and nonmutator agents.
      In bottom panel, \ref{fig:spatial-structure:denovo}, mutators are initially absent and arise spontaneously \textit{de novo} with probability $10^{-6}$.
      Note that $y$-axis scale differs substantially between top and bottom panels.
      Effects of population structure are more pronounced under \textit{de novo} conditions, when initial supply of mutators is scarce.
      Under well-mixed conditions, increase in population size favors mutators; nonmutators reliably persist in very large populations through fewer than 10 available beneficial mutations.
      In contrast, under 1D deme structure, very large population sizes are favorable to nonmutators.
      With such strong spatial structure, very large populations of nonmutators reliably resist mutator fixation even when upwards of 20 beneficial mutations are available.
      Results are under GPU simulation, using counter-based genome model.
      In lower panel, some treatments with smallest population size were unable to reliably fix all available beneficial mutaitons within allotted runtime; panel areas in these regimes are left blank.
      Supplementary Figures \cref{fig:fixheat-5050-cupy,fig:fixheat-denovo-cupy} detail results in a tabular format.
    }
    \label{fig:spatial-structure}
  \end{minipage}
\end{figure}


Notably, this result is a different story than was found by \citet{raynes2019migration} when studying effect of spatial structure on hypermutatoe prevalence.
Whereas
