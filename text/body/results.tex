\section{Results and Discussion} \label{sec:results}

In this work, we investigate the interactions between population size, adaptive potential, and population structure in influencing selection on mutator traits within asexual populations.
In our first experiments, we replicate previous work by \citet{raynes2018sign} showing sign-change effects of population size on mutator favorability.
Subsequent experiments investigate how these dynamics change when limited beneficial mutations are available and their sensitivity to initial abundance of mutators in the population.
Finally, experiments survey behavior across three regimes of population spatial structure.
We find that the sign effect of population size on mutator favorability is dependent on spatial structure.

\subsection{Scaling Population Size}
\label{sec:scaling-population-size}

In a first set of experiments, we tested the capability of our model to reproduce the qualitative regimes of mutator dynamics across population scales identified by \citet{raynes2018sign}.
In the first regime, where population size is very small, fixation of mutator and nonmutator alleles is nearly equiprobable, owing to the overwhelming influence of stochastic effects.
Subsequently, in medium-sized populations, mutators become disfavored on account of increased power of selection becoming in penalizing the increased mutational load imposed by mutator allelees.
Finally, in sufficiently large populations, a sign-change fitness effect occurs and mutator fixation becomes overwhelmingly favored.
This effect occurs in asexual populations because population size increases the probability of at least one mutator discovering a beneficial mutation, and so ``hitch-hiking'' to sweep out nonmutators.

\begin{figure}[h]
  % adapted from https://tex.stackexchange.com/a/122813/316176
  \captionsetup[subfigure]{justification=raggedright}
  \begin{minipage}{0.7\textwidth}

    \begin{minipage}{0.04\textwidth}~\end{minipage}%
    \begin{minipage}{0.44\textwidth}
      \centering
      \itshape
      \textbf{unlimited} beneficial mutations
    \end{minipage}%
    \begin{minipage}{0.34\textwidth}
      \centering
      \itshape
      \textbf{one} beneficial mutation
    \end{minipage}

    ~\vspace{-1.5ex}

    % Top subfigure
    \begin{subfigure}[b]{\linewidth}
      \begin{minipage}{0.88\textwidth}
        \begin{minipage}{0.53\textwidth}
          \includegraphics[height=2.9cm, trim={0cm 0.2cm 3.9cm 0.8cm}, clip]{binder/binder-wse-5050-spatial2d-32atile-infben-traits.ipynb/binder/teeplots/wse-5050-spatial2d-32atile-infben-traits/errorbar=ci+hue=genotype+num-abm=(inf,)+style=genotype+viz=size-fixation-areaplot+x=population-size+y=fixation-probability+ext=.pdf}%
        \end{minipage}%
        \begin{minipage}{0.47\textwidth}
          \includegraphics[height=2.9cm, trim={1.3cm 0.2cm 3.9cm 0.8cm}, clip]{binder/binder-wse-5050-spatial2d-32atile-infben-traits.ipynb/binder/teeplots/wse-5050-spatial2d-32atile-infben-traits/errorbar=ci+hue=genotype+num-abm=(1.0,)+style=genotype+viz=size-fixation-areaplot+x=population-size+y=fixation-probability+ext=.pdf}
        \end{minipage}
      \end{minipage}%
      \hspace{-3ex}%
      \begin{minipage}{0.12\textwidth}
        \raggedright
        \caption{\footnotesize 32 agents per PE\\~\\~\\}
        \label{fig:wse-inf-one:32}
      \end{minipage}%
    \end{subfigure}%

    % Bottom subfigure - Adjusted layout to match the top
    \begin{subfigure}[b]{\linewidth}
      \begin{minipage}{0.88\textwidth}
        \begin{minipage}{0.53\textwidth}
          \includegraphics[height=2.9cm, trim={0cm 0.2cm 3.9cm 0.8cm}, clip]{binder/binder-wse-5050-spatial2d-2048atile-infben-traits.ipynb/binder/teeplots/wse-5050-spatial2d-2048atile-infben-traits/errorbar=ci+hue=genotype+num-abm=(inf,)+style=genotype+viz=size-fixation-areaplot+x=population-size+y=fixation-probability+ext=.pdf}%
        \end{minipage}%
        \begin{minipage}{0.47\textwidth}
          \includegraphics[height=2.9cm, trim={1.3cm 0.2cm 3.9cm 0.8cm}, clip]{binder/binder-wse-5050-spatial2d-2048atile-infben-traits.ipynb/binder/teeplots/wse-5050-spatial2d-2048atile-infben-traits/errorbar=ci+hue=genotype+num-abm=(1.0,)+style=genotype+viz=size-fixation-areaplot+x=population-size+y=fixation-probability+ext=.pdf}
        \end{minipage}
      \end{minipage}%
      \hspace{-3ex}%
      \begin{minipage}{0.1\textwidth}
        \caption{\footnotesize 2,048 agents per PE\\~\\~\\}
        \label{fig:wse-inf-one:2048}
      \end{minipage}%
    \end{subfigure}%

~\vspace{-1.2ex}

\includegraphics[width=\textwidth, trim={0cm 4.7cm 5.3cm 0cm}, clip]{binder/binder-wse-5050-spatial2d-2048atile-infben-traits.ipynb/binder/teeplots/wse-5050-spatial2d-2048atile-infben-traits/col=available-beneficial-mutations+errorbar=sd+hue=genotype+kind=line+style=genotype+viz=relplot+x=population-size+y=fixation-probability+ext=}

  \end{minipage}%
  \begin{minipage}{0.3\textwidth}
    \caption{%
      \textbf{Restricted beneficial mutation supply favors normomutators in large populations.}
      \footnotesize
      TODO, 50/50 start conditions on WSE.
    }
    \label{fig:wse-inf-one}
  \end{minipage}
\end{figure}


As shown in the left panel of Figure \ref{fig:wse-inf-one:32}, our simulations reproduce the sign-change fitness effect described by \citet{raynes2018sign}.
Leveraging the capabilites of WSE accelerator hardware, we tested population sizes several orders of magnitude beyond those reported in \citet{raynes2018sign}.
For these experiments, 2D population structure was used with 50/50 initialization.
In line with expectations, no further qualitative differences in mutator fixation probability emerged in these larger population sizes, which ranged up to 1 billion agents.
For the largest-scale trials, shown left panel of Figure \ref{fig:wse-inf-one:2048}, an increased subpopulation sizes 2,048 agents per deme was used.
As such, in these trials, note that coarsened spacing between surveyed population sizes obscures the sign-change fitness effect near the lower bound of population size.

\subsection{Restricting Adaptive Potential}
\label{sec:restricting-adaptive-potential}

Next, we sought to establish how adaptive potential interacts with current synthesis of the role of population size in mediating mutator dynamics.
One assumption of existing work by \citet{raynes2018sign} is availability of abundant potential beneficial mutations.
Existing simulation studies suggest restricting adaptive potential should disfavor mutator alleles \citep{tenaillon1999mutators}, but understanding of how this effect interacts with other factors is limited.

To investigate this question, we next performed experiments where the supply of possible beneficial mutations was constrained.
As for the above results, experiments were conducted on the WSE platform using 2D population structure with 50/50 initialization.
The right panels of \cref{fig:wse-inf-one:32,fig:wse-inf-one:2048} show mutator fixation probabilities where agents could accrue no more than one beneficial mutation, for trials usig 32- and 2,048 agents per deme respectively.
In both cases, a second sign-change fitness effect for mutator traits appears, with nonmutators regaining favor past population sizes of 100,000 agents.
Beyond population sizes of 1 million agents, nonmutators reliably drive mutators to extinction.

These results substantiate an earlierverbal hypothesis by \citet{desai2011balance} that more than one beneficial mutation should become necessary to fix a mutator allele when fixation time becomes long, as is the case with large population size.
Rationale behind this expectation is that where fixation time is long, nonmutators become likely to also discover a comparable beneficial allele before being driven to extinction.
From this point, disadvantage in mutational load dooms the mutator allele.

\begin{figure*}

\begin{minipage}{0.65\textwidth}
  \includegraphics[width=\textwidth]{binder/binder-wse-5050-spatial2d-2048atile-traits.ipynb/binder/teeplots/wse-5050-spatial2d-2048atile-traits/col=available-beneficial-mutations+errorbar=ci+hue=genotype+style=genotype+viz=size-fixation-areaplot+x=population-size+y=fixation-probability+ext=.pdf}%
  \end{minipage}
\begin{minipage}{0.3\textwidth}
\caption{
\textbf{Availability of adaptive potential shifts favor from normomutators to hypermutators.}
\footnotesize
Area plots show normo- vs. hypermutator fixation probabilities.
As available beneficial mutations are increased, hypermutators gain favor in progressively larger population sizes.
Simulations were conducted on WSE using the Poisson genome model, with populations initialized to a 50/50 mix of normo- and hypermutators.
Subpopulations comprised 2,048 agents per PE.
Error bands show bootstrapped 95\% confidence intervals.
}
\end{minipage}

\end{figure*}


Next, we tested sensitivity of the relationship between mutator fixation probability and population size with respect to the amount of adaptive potential available.
Figure \ref{fig:avail-ben-muts} shows fixation curves where one, two, three, four, and five beneficial mutations are available, conducted with 256 agents per deme.
Under surveyed conditions, with 2D spatial structure and mutators initially in equal proportion to nonmutators, nonmutator persistence rapidly decays with increases in adaptive potential.
Even with the largest-surveyed billion-agent population size, mutators regain selective parity where four or more beneficial mutations are available.
Interestingly, this result may help explain an earlier observation by \citet{tenaillon1999mutators} that below a threshold of four beneficial mutations available, mutator alleles never rose from background levels to fix within their compartment-based model of a well-mixed billion-member population.
Notwithstanding differences in model parameterization and population structure, our findings suggest thi threshold may coincide with the point where the net fitness effect of the mutator allele neutralizes in commensurately-scaled populations.

\begin{figure*}

\begin{minipage}{0.65\textwidth}
  \includegraphics[width=\textwidth]{binder/binder-cupy-5050-traits.ipynb/binder/teeplots/cupy-5050-traits/col=available-beneficial-mutations+errorbar=ci+hue=population-structure+kind=line+palette=dark2+style=population-structure+viz=relplot+x=population-size+y=fixation-probability+ext=.pdf}%
\end{minipage}
\begin{minipage}{0.3\textwidth}
\caption{
\textbf{Well-mixed populations experience similar effects of adaptive potential on normalized fixation probability.}
\footnotesize
In both scenarios, as available beneficial mutations are increased, mutators gain favor in progressively larger population sizes.
Simulations were conducted on GPU using the counter-based genome model, with populations initialized to a 50/50 mix of non- and mutators.
Subpopulations comprised 256 agents per PE.
Error bands show bootstrapped 95\% confidence intervals.
}
\label{fig:avail-ben-muts-gen}
\end{minipage}

\end{figure*}


Given that experiments conducted on the WSE platform all involve 2D spatial structure, we implemented an additional set of GPU-based experiments to compare findings with well-mixed population structure.
Figure \ref{fig:avail-ben-muts-gen} overviews these results.
Although experments ranged up to a smaller maximum population size on the GPU platform, detected regimes of mutator dynamics are consistent with WSE-based experiments.
Comparing on the basis of spatial structure, we find fixation probabilities to be near-identical between GPU-based trials using 2D and well-mixed structure across surveyed conditions.
Corrected for multiple comparisons using the Holm-Bonferroni method, we find significant divergence in fixation probability only for TODO (TODO).
% https://github.com/mmore500/hypermutator-dynamics/issues/38

\subsection{Effects of Background Mutator Prevalence}
\label{sec:background-hypermutator-prevalence}

\begin{figure}[h]
\begin{minipage}{0.6\textwidth}
  \includegraphics[width=\linewidth]{binder/binder-wse-traits.ipynb/binder/teeplots/wse-traits/errorbar=ci+hue=initial-conditions+kind=line+palette=set2-r+row=population-size+style=initial-conditions+viz=relplot+what=counter-based+x=available-beneficial-mutations+y=fix-prob+ext=.pdf}%
\end{minipage}%
\begin{minipage}{0.4\textwidth}
  \caption{
  \textbf{Initial supply of mutators influences nonmutator resilience to available adaptive potential.}
  \footnotesize
  Lineplot strips show change in mutator fixation probability with adaptive potential for surveyed population sizes, with error bands providing bootstrapped 95\% confidence intervals.
  Simulations were conducted on WSE, using a population size of 256 agents per PE and counter-based genome model.
  Results using a site-explicit model were similar, provided in Supplementary \cref{fig:denovo-5050-conditions-combined-site-explicit,fig:wse-site-explicit-counter-based}.
  Supplementary Figure \ref{fig:fixheat-wse-256atile} details results in a tabular format.
    }
    \label{fig:denovo-5050-conditions-combined}
  \end{minipage}
\end{figure}


Having observed large population size prevent mutator fixation in competition experiments under conditions of limited adaptive potential, we next sought to assess how such an effect might manifest in more naturalistic conditions.
For this purpose, we relaxed the initial condition that mutators begin ``50/50'' in equal proportion to nonmutators.
Instead, we allowed mutators to enter the population at a low background rate \citep{desai2011balance,johnson1999approach}, an experimental design common with several previous simulation studies of hypermutator dynamics \citep{wylie2009fixation,tenaillon1999mutators}.
To conservatively TODO
We selected a relatively conservative introduction probability of $10^{-6}$.

Figure \ref{fig:denovo-5050-conditions-combined} compares mutator fixation probabilities between 50/50 and \textit{de novo} conditions.
Trials were conducted with 2D spatial structure on the WSE platform.
With 256 agents per deme, population size scaled up to 136 million agents.

Prior work analyzing conditions with infinite adaptive potential has shown per-capita favorability of mutator traits to be density independent \citep{raynes2019selection}.
As such, a lower background rate for mutator alleles should be expected to substantially reduce the net probability of mutator fixation.
In line with expectations, the \textit{de novo} treatment substantially reduces mutator fixation.
Under \textit{de novo} conditions, nonmutators reliably resist mutator fixation in populations of 136 million through more than 10 beneficial mutations available.
Under 50/50 conditions, by contrast, mutators begin to fix in these large populations past 2 beneficial mutations available.

We performed sensitivity analysis of our findings to the underlying counter-based genome model used.
For this purpose, we implemented an alternate ``site-explicit'' model where the probability of discovering a beneficial mutation scales proportional to the number available.
As shown in Supplementary Figure \ref{fig:wse-site-explicit-counter-based}, results were generally consistent between the two genome models.
Notably, though, under the ``site-explicit'' model, mutator alleles more frequently fixed within the smallest surveyed population size under 50/50 initialization.

\subsection{Spatiotemporal Dynamics}
\label{sec:dynamics}

To assess how competition between non- and mutator strains unfolds at very large population scales, we performed an additional set of simulations on the WSE using the site-explicit model of adaptation, where mutators originated \textit{de novo} from nonmutator strains.
For these experiments, a subpopulation size of 256 agents per PE was used to preserve on-device memory to record time series data.
This configuration yielded a net population size of 191 million agents.
In these trials, we probed the critical region where outcomes transition between nonmutator and mutator fixation, configuring adaptive potential between 12 and 16 available beneficial mutations.

\begin{figure}[ht]
  \centering
  %----------------------------
  % First row: NO FIXATION
  \begin{subfigure}[t]{\textwidth}
    \centering
    % Left minipage for row label (empty caption)
    \begin{minipage}[b]{0.04\textwidth}
      \caption{}
      \label{fig:dynamics:no-fixation}
    \end{minipage}
    % Six images (columns)
    \begin{minipage}[b]{0.15\textwidth}
      \includegraphics[width=\linewidth]{binder/binder-wse-denovo-spatial2d-explicitsite-timeseries.ipynb/binder/wse-denovo-spatial2d-explicitsite-timeseries/a=traitframes+nmut=14+rep=39a89ca6-a1b5-4b32-ae5f-f0dbb40ba027/dstream_Tbar=000504+ext=.png}
    \end{minipage}
    \begin{minipage}[b]{0.15\textwidth}
      \includegraphics[width=\linewidth]{binder/binder-wse-denovo-spatial2d-explicitsite-timeseries.ipynb/binder/wse-denovo-spatial2d-explicitsite-timeseries/a=traitframes+nmut=14+rep=5dc8e084-0382-4d7b-9b76-6c3902ca3c1d/dstream_Tbar=000952+ext=.png}
    \end{minipage}
    \begin{minipage}[b]{0.15\textwidth}
      \includegraphics[width=\linewidth]{binder/binder-wse-denovo-spatial2d-explicitsite-timeseries.ipynb/binder/wse-denovo-spatial2d-explicitsite-timeseries/a=traitframes+nmut=14+rep=5dc8e084-0382-4d7b-9b76-6c3902ca3c1d/dstream_Tbar=001400+ext=.png}
    \end{minipage}
    \begin{minipage}[b]{0.15\textwidth}
      \includegraphics[width=\linewidth]{binder/binder-wse-denovo-spatial2d-explicitsite-timeseries.ipynb/binder/wse-denovo-spatial2d-explicitsite-timeseries/a=traitframes+nmut=14+rep=5dc8e084-0382-4d7b-9b76-6c3902ca3c1d/dstream_Tbar=002552+ext=.png}
    \end{minipage}
    \begin{minipage}[b]{0.15\textwidth}
      \includegraphics[width=\linewidth]{binder/binder-wse-denovo-spatial2d-explicitsite-timeseries.ipynb/binder/wse-denovo-spatial2d-explicitsite-timeseries/a=traitframes+nmut=14+rep=5dc8e084-0382-4d7b-9b76-6c3902ca3c1d/dstream_Tbar=020472+ext=.png}
    \end{minipage}
    \begin{minipage}[b]{0.15\textwidth}
      \includegraphics[width=\linewidth]{binder/binder-wse-denovo-spatial2d-explicitsite-timeseries.ipynb/binder/wse-denovo-spatial2d-explicitsite-timeseries/a=traitframes+nmut=14+rep=5dc8e084-0382-4d7b-9b76-6c3902ca3c1d/dstream_Tbar=061424+ext=.png}
    \end{minipage}
  \end{subfigure}

  \vspace{1em} % vertical space between rows

  %----------------------------
  % Second row: FIXATION
  \begin{subfigure}[t]{\textwidth}
    \centering
    % Left minipage for row label (empty caption)
    \begin{minipage}[b]{0.05\textwidth}
      \caption{}
      \label{fig:dynamics:fixation}
    \end{minipage}
    % Six images (columns)
    \begin{minipage}[b]{0.15\textwidth}
      \includegraphics[width=\linewidth]{binder/binder-wse-denovo-spatial2d-explicitsite-timeseries.ipynb/binder/wse-denovo-spatial2d-explicitsite-timeseries/a=traitframes+nmut=14+rep=39a89ca6-a1b5-4b32-ae5f-f0dbb40ba027/dstream_Tbar=000504+ext=.png}
    \end{minipage}
    \begin{minipage}[b]{0.15\textwidth}
      \includegraphics[width=\linewidth]{binder/binder-wse-denovo-spatial2d-explicitsite-timeseries.ipynb/binder/wse-denovo-spatial2d-explicitsite-timeseries/a=traitframes+nmut=14+rep=39a89ca6-a1b5-4b32-ae5f-f0dbb40ba027/dstream_Tbar=000952+ext=.png}
    \end{minipage}
    \begin{minipage}[b]{0.15\textwidth}
      \includegraphics[width=\linewidth]{binder/binder-wse-denovo-spatial2d-explicitsite-timeseries.ipynb/binder/wse-denovo-spatial2d-explicitsite-timeseries/a=traitframes+nmut=14+rep=39a89ca6-a1b5-4b32-ae5f-f0dbb40ba027/dstream_Tbar=001400+ext=.png}
    \end{minipage}
    \begin{minipage}[b]{0.15\textwidth}
      \includegraphics[width=\linewidth]{binder/binder-wse-denovo-spatial2d-explicitsite-timeseries.ipynb/binder/wse-denovo-spatial2d-explicitsite-timeseries/a=traitframes+nmut=14+rep=39a89ca6-a1b5-4b32-ae5f-f0dbb40ba027/dstream_Tbar=002489+ext=.png}
    \end{minipage}
    \begin{minipage}[b]{0.15\textwidth}
      \includegraphics[width=\linewidth]{binder/binder-wse-denovo-spatial2d-explicitsite-timeseries.ipynb/binder/wse-denovo-spatial2d-explicitsite-timeseries/a=traitframes+nmut=14+rep=39a89ca6-a1b5-4b32-ae5f-f0dbb40ba027/dstream_Tbar=002489+ext=.png}
    \end{minipage}
    \begin{minipage}[b]{0.15\textwidth}
      \includegraphics[width=\linewidth]{binder/binder-wse-denovo-spatial2d-explicitsite-timeseries.ipynb/binder/wse-denovo-spatial2d-explicitsite-timeseries/a=traitframes+nmut=14+rep=39a89ca6-a1b5-4b32-ae5f-f0dbb40ba027/dstream_Tbar=002489+ext=.png}
    \end{minipage}
  \end{subfigure}

  %----------------------------
  % Timepoint labels below the bottom row:
  \begin{minipage}[c]{0.154\textwidth}
\hfill
\begin{varwidth}{\textwidth}
$T = 504$
\end{varwidth}
\hfill
  \end{minipage}
  \begin{minipage}[c]{0.154\textwidth}
\hfill
\begin{varwidth}{\textwidth}
$T = 952$
\end{varwidth}
\hfill
  \end{minipage}
  \begin{minipage}[c]{0.154\textwidth}
\hfill
\begin{varwidth}{\textwidth}
$T = 1,400$
\end{varwidth}
\hfill
  \end{minipage}
  \begin{minipage}[c]{0.154\textwidth}
\hfill
\begin{varwidth}{\textwidth}
$T = 2,552$
\end{varwidth}
\hfill
  \end{minipage}
  \begin{minipage}[c]{0.154\textwidth}
\hfill
\begin{varwidth}{\textwidth}
$T = 20,472$
\end{varwidth}
\hfill
  \end{minipage}
  \begin{minipage}[c]{0.154\textwidth}
\hfill
\begin{varwidth}{\textwidth}
$T = 61,424$
\end{varwidth}
\hfill
  \end{minipage}
  \caption{
  \textbf{Spatiotemporal composition of simulated populations from Wafer-Scale Engine experiments.}
  \footnotesize
  Snapshots show 191 million agent populations with 2D spatial structure, using site-explicit genome model configured to adaptive potential of 14 beneficial mutations available and hypermutators introduced \textit{de novo}.
  Raster values are binary, with white pixels indicating a sampled normomutators and black pixels indicating a sampled hypermutator.
  Subpanels \ref{fig:dynamics:no-fixation} and       \ref{fig:dynamics:fixation} show replicates where hypermutators do not, and do, reach fixation, respectively.
  Animations are provided at \url{https://hopth.ru/ej} and \url{https://hopth.ru/ek}.
  Example timecourses for simulations with 12 and 16 beneficial mutations available can be found at \url{https://hopth.ru/el} and \url{https://hopth.ru/em}.
  }
  \label{fig:dynamics}
\end{figure}


Figure \ref{fig:dynamics} arranges sequential population snapshots from two example simulations, both with adaptive potential of 14 beneficial mutations.
In the first example, mutators fixed; in the second, they did not.
At outset, panels \ref{fig:dynamics:no-fixation} and \ref{fig:dynamics:fixation} unfold similarly.
First, mutator strains appear \textit{de novo} across the breadth of simulated populations.
Then, as patches of mutators expand outwards, the nonmutator population is broken into pockets.
Mutator strains continued to stem from nonmutator lineages as they accrue adaptive mutations.
Of 16 replicates where we observed mutators fix, we found 2 cases where the final dominant mutator lineage arose from a partially-adapted nonmutator lineage --- harboring 1 and 3 beneficial mutations, respectively.
% https://github.com/mmore500/hypermutator-dynamics/blob/90422b30f2dd9ca29baad273cbcc676dbfeab55f/binder/wse-denovo-spatial2d-explicitsite-genomes.ipynb
In scenarios where mutators fix, surviving patches of nonmutators dwindle away entirely (panel \ref{fig:dynamics:fixation}).
Where nonmutators persist, they re-establish via concentric growth from surviving pockets (panel \ref{fig:dynamics:no-fixation}).

\begin{figure}

\begin{minipage}{0.4\textwidth}
\includegraphics[width=\textwidth]{binder/binder-wse-denovo-spatial2d-explicitsite-timeseries-2025-02-07-peaksweep.ipynb/binder/teeplots/wse-denovo-spatial2d-explicitsite-timeseries-2025-02-07-peaksweep/color=orange+kind=line+viz=relplot+x=cerebraslib-hypermut-num-avail-ben-muts+y=dstream-value+ext=.pdf}
\end{minipage}
\begin{minipage}{0.55\textwidth}
\caption{
\textbf{Mutators prevalence transitively peaks above population majority in scenarios with limited adaptive potential.}
\footnotesize
Simulations were conducted on WSE, configured with population size 191 million over 2D spatial structure and site-explicit genome model with mutators introduced \textit{de novo}.
Under these conditions, mutator fixation was observed only for 14 or more available beneficial mutations.
Error bands show 100\% percentile interval across four replicates.
}
\label{fig:peaksweep}
\end{minipage}
\end{figure}


At peak mutator prevalence, we observed nonmutators shrink to less than 0.1\% of the population in experiments with adaptive potential of 14 available beneficial mutations.
% ^^^ https://github.com/mmore500/hypermutator-dynamics/blob/5e904ac2fbbd4b41a0ad679883fc1a63af71c00c/binder/wse-denovo-spatial2d-explicitsite-timeseries.ipynb
To assess the magnitude of transient peak mutator concentrations across levels of adaptive potential, we performed an additional set of four-replicate simulations ranging from adaptive potentials of 2 to 10 available beneficial mutations.
For all replicates with 6 or more available beneficial mutations, mutators peaked beyond 50\% prevalence (Figure \ref{fig:peaksweep}).

\subsection{Effects of Population Structure}
\label{sec:population-structure}

In a final set of experiments, we sought to establish the influence of spatial structure on fixation outcomes to further assess how robustly natural asexual populations might resist mutator fixation in scenarios with limited adaptation potential.

The simulation experiments described above, conducted on WSE hardware, all assumed a global population structure as an interconnected 2D grid of smaller, well-mixed constituent demes.
To explore this dimension, we conducted trials with the default 2D structure along with (1) a treatment arranging demes in a 1D ring and (2) a fully well-mixed treatment with no subpopulation structure.
Because work reported in this section used a GPU platform, rather than WSE, population sizes ranged only up to 15 million agents.

\begin{figure}[h]
\begin{minipage}{\textwidth}
  \includegraphics[width=\linewidth]{binder/binder-cupy-traits.ipynb/binder/teeplots/cupy-traits/errorbar=ci+exclude=1D-demes+hue=population-structure+kind=line+palette=dark2+row=population-size+style=initial-conditions+viz=relplot+x=available-beneficial-mutations+y=fix-prob+ext=.pdf}%
\end{minipage}

\begin{minipage}{\textwidth}
  \caption{%
    \textbf{Population structure increases normomutator resilience to available adaptive potential.}
    \footnotesize
    Lineplot strips show relationship between adaptive potential and hypermutator fixation probability across surveyed population sizes, with error bands indicating bootstrapped 95\% confidence intervals.
    In the 50/50 treatment, experiments were initialized with an even mix of hyper- and normomutator agents.
    In the \textit{de novo} treatment, hypermutators are initially absent and arise spontaneously \textit{de novo} with probability $10^{-6}$.
    Effects of population structure are more pronounced under \textit{de novo} conditions, when initial supply of hypermutators is scarce.
    Under well-mixed conditions, increase in population size favors hypermutators; normomutators reliably persist in very large populations through fewer than 10 available beneficial mutations.
    In contrast, under 2D deme structure, very large population sizes are favorable to normomutators.
    With such strong spatial structure, very large populations of normomutators resist hypermutator fixation even when upwards of 10 beneficial mutations are available.
    Results are under GPU simulation, using counter-based genome model.
    Supplementary Figures \cref{fig:fixheat-5050-cupy,fig:fixheat-denovo-cupy} detail results in a tabular format.
  }
  \label{fig:spatial-structure-combined}
\end{minipage}
\end{figure}


Figure \ref{fig:spatial-structure} plots increase in mutator fixation probabilities with available adaptive potential across a spectrum of population sizes.
Results differ substantially between 50/50 and \textit{de novo} conditions, appearing in panels \ref{fig:spatial-structure:5050} and \ref{fig:spatial-structure:denovo} respectively.
With 50/50 initializations, nonmutators reliably persist through only one or two beneficial mutations available across all spatial configurations.
By contrast, under \textit{de novo} conditions, nonmutator populations with 1D structure successfully resist mutator invasion through upwards of 20 beneficial mutations.
Notably, though, under weak, well-mixed spatial structure, the effect of population size is opposite --- rather than suppressing mutator favorability, mutators actually become \textit{more} likely to sweep large populations.

These findings contrast with previous modeling work assuming unlimited adaptive potential, which predicts that so long as any level of connectivity exists between subpopulations, spatial structure boosts mutator fixation probability \citep{raynes2019migration}.
One possible explanation for why spatial structure can decrease mutator fixation probability is by extending the window of opportunity for nonmutators to catch up in exploiting available adaptive potential before being driven to extinction.
Such a possibility aligns with findings from \textit{in vivo} evolution experiments with \textit{E. coli} that show that migration barriers delaying introduction of mutator strains into nonmutator populations can reduce mutator advantage \citep{lechat2006escherichia}.
