\begin{abstract}
Mutator alleles, which elevate mutation rates, introduce a fundamental trade-off in asexual organisms by accelerating adaptation but also increasing mutation load.
While large population sizes generally favor the fixation of mutators due to increased opportunity for beneficial mutation discovery, the reasons some large asexual populations resist mutator fixation are less understood.
In this study, we investigate how scenarios with few beneficial mutations available (i.e., limited adaptive potential) affect mutator dynamics across a continuum of population sizes and spatial structures.
Experiments utilizing stochastic agent-based simulations show that when adaptive potential is restricted, very large populations favor nonmutators --- an inversion of the trend observed under unlimited adaptive potential.
Moreover, we find that strong spatial structure suppresses mutator fixation within large populations when the mutator allele is initially rare and adaptive potential is limited.
In contrast, under equivalent conditions with well-mixed structure, large population size has an opposite effect --- instead promoting mutator fixation.
These findings underscore the critical role of adaptive potential in shaping the evolution of mutation rates in asexual populations, with implications for understanding and managing hypermutation-associated phenomena in both laboratory and real-world contexts.
We also describe technical aspects of the work in harnessing the 850,000 processor Cerebras Wafer-scale Engine to execute simulations, which allowed up to a $111{,}091\times$ speedup over single-core CPU execution.
\end{abstract}
