\begin{figure*}

\begin{minipage}{0.65\textwidth}
  \includegraphics[width=\textwidth]{binder/binder-wse-5050-spatial2d-2048atile-traits.ipynb/binder/teeplots/wse-5050-spatial2d-2048atile-traits/col=available-beneficial-mutations+errorbar=ci+hue=genotype+style=genotype+viz=size-fixation-areaplot+x=population-size+y=fixation-probability+ext=.pdf}%
  \end{minipage}
\begin{minipage}{0.3\textwidth}
\caption{
\textbf{Availability of adaptive potential shifts favor from normomutators to hypermutators.}
\footnotesize
Area plots show normo- vs. hypermutator fixation probabilities.
As available beneficial mutations are increased, hypermutators gain favor in progressively larger population sizes.
Simulations were conducted on WSE using the counter-based genome model, with populations initialized to a 50/50 mix of normo- and hypermutators.
Subpopulations comprised 2,048 agents per PE.
Error bands show bootstrapped 95\% confidence intervals.
Supplementary Figure \ref{fig:fixheat-wse-altatile:2048} details results in a tabular format.
}
\label{fig:avail-ben-muts}
\end{minipage}

\end{figure*}
