\begin{figure}[h]
\begin{minipage}{\textwidth}
  \includegraphics[width=\linewidth]{binder/binder-cupy-traits.ipynb/binder/teeplots/cupy-traits/errorbar=ci+exclude=1D-demes+hue=population-structure+kind=line+palette=dark2+row=population-size+style=initial-conditions+viz=relplot+x=available-beneficial-mutations+y=fix-prob+ext=.pdf}%
\end{minipage}

\begin{minipage}{\textwidth}
  \caption{%
    \textbf{Population structure increases normomutator resilience to available adaptive potential.}
    \footnotesize
    Lineplot strips show relationship between adaptive potential and hypermutator fixation probability across surveyed population sizes, with error bands indicating bootstrapped 95\% confidence intervals.
    In the 50/50 treatment, experiments were initialized with an even mix of hyper- and normomutator agents.
    In the \textit{de novo} treatment, hypermutators are initially absent and arise spontaneously \textit{de novo} with probability $10^{-6}$.
    Effects of population structure are more pronounced under \textit{de novo} conditions, when initial supply of hypermutators is scarce.
    Under well-mixed conditions, increase in population size favors hypermutators; normomutators reliably persist in very large populations through fewer than 10 available beneficial mutations.
    In contrast, under 2D deme structure, very large population sizes are favorable to normomutators.
    With such strong spatial structure, very large populations of normomutators resist hypermutator fixation even when upwards of 10 beneficial mutations are available.
    Results are under GPU simulation, using counter-based genome model.
    Supplementary Figures \cref{fig:fixheat-5050-cupy,fig:fixheat-denovo-cupy} detail results in a tabular format.
  }
  \label{fig:spatial-structure-combined}
\end{minipage}
\end{figure}
