\begin{sidewaysfigure}[h]
\begin{subfigure}{0.5\linewidth}
  \includegraphics[width=\linewidth]{binder/binder-wse-traits.ipynb/binder/teeplots/wse-traits/errorbar=ci+hue=genome-model+kind=line+palette=accent+row=population-size+style=genome-model+viz=relplot+what=50-50+x=available-beneficial-mutations+y=fix-prob+ext=.pdf}%
  \caption{50/50 initial conditions}
\end{subfigure}%
\begin{subfigure}{0.5\linewidth}
  \includegraphics[width=\linewidth]{binder/binder-wse-traits.ipynb/binder/teeplots/wse-traits/errorbar=ci+hue=genome-model+kind=line+palette=accent+row=population-size+style=genome-model+viz=relplot+what=de-novo+x=available-beneficial-mutations+y=fix-prob+ext=.pdf}%
  \caption{\textit{de novo} initial conditions}
\end{subfigure}

\begin{minipage}{\textwidth}
  \caption{
  \textbf{Comparison beteween site-explicit and counter-based genome model.}
  \footnotesize
  Lineplot strips track mutator fixation probability across surveyed levels of adaptive potential, with error bands providing bootstrapped 95\% confidence intervals.
  Simulations were conducted on WSE, using a population size of 256 agents per PE.
  Supplementary Figure \ref{fig:fixheat-wse-256atile} details results in a tabular format.
    }
    \label{fig:wse-site-explicit-counter-based}
  \end{minipage}
\end{sidewaysfigure}
