\begin{figure*}

\begin{subfigure}{0.5\textwidth}
\includegraphics[width=\textwidth, trim={0cm 0cm 2.5cm 0cm}, clip]{binder/binder-perf-wse-vs-gpu.ipynb/binder/teeplots/col=experiment-design+hue=hardware+orient=h+viz=backplot+x=throughput-agent-generations-sec+ext=.pdf}
\caption{throughput}
\label{fig:perf:throughput}
\end{subfigure}%
\hfill
\begin{subfigure}{0.49\textwidth}
\includegraphics[width=\textwidth]{binder/binder-perf-wse-vs-gpu.ipynb/binder/teeplots/hue=hardware+orient=h+viz=backplot+x=speedup+y=experiment-design+ext=.pdf}

~
\caption{speedup}
\label{fig:perf:speedup}
\end{subfigure}

\caption{
  \textbf{Runtime performance for CPU, GPU, and WSE simulations.}
  \footnotesize
  CPU throughput was measured from NumPy-backed simulation code, GPU throughput was measured from equivalent CuPy-backed simulation code, and WSE throughput was measured from a comparable Cerebras Software Language implementation.
  CPU experiments were single-core on AMD EPYC 7H12 (2.595 GHz), GPU hardware was NVIDIA A100 hosted by AMD EPYC 7713 (2.0 GHz) or Intel Xeon 8358 (2.6 GHz), and WSE hardware was CS-2 hosted by Intel Xeon Platinum 8280L (2.7-4.0GHz).
  Speedup was calculated relative to mean CPU throughput.
  Shaded areas indicate bootstrapped 95\% confidence interval with sample size $\geq 10$.
}
\label{fig:perf}
\end{figure*}
